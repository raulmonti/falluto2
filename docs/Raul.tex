\documentclass[titlepage, 10pt]{article}


\pagestyle{plain}
\title{Revisi\'{o}n y extensi\'{o}n de un model checker para la vericaci\'{o}n autom\'{a}tica de sistemas tolerantes a fallas}
\author{Raul Monti}
\date{jue 6 de sep, 2012}



\begin{document}

\maketitle

\tableofcontents

\newpage
\section{Introducci\'on al model checking}
\label{introduction}
\subsection{\textbf{NuSMV}}

NuSMV\cite{NuSMVBibl} es un model checker simb\'olico originado en la reingenieria, reimplementaci\'on
y extensi\'on of CMU SMV. Usamos su segunda versi\'on la cual implementa tanto t\'ecnicas
de \textbf{model checking simb\'olico basado en BDD}, como  t\'ecnicas de model checking
basadas en \textbf{satisfactibilidad proposisional (SAT)}.
Cada una de estas t\'ecnicas sulen ser \'utiles para resolver diferentes tipos
de problemas y por lo tanto pueden considerarse complementarias.\\

El c\'odigo de \emph{NuSMV} se distribuye bajo licencia LGPL-2.1 la cual permite el 
uso gratuito del software por parte de cualquier persona interesada asi como a 
contribuir con el desarrollo del mismo. La adopci\'on de esta licencia ha permitido la
contribuci\'on al proyecto por parte de numerosos \'ambitos cient\'ificos asi como
industriales.\\

El proyecto de software libre NuSMV apunta a crear una base com\'un para
el estudio y comparaci\'on de diferentes t\'ecnicas de model checkin simb\'olico.\\

\begin{itemize}


\item%
\textbf{Funcionalidades:}\\

\emph{NUSMV} permite la representaci\'on de sistemas finitos tanto s\'incronos
como as\'incronos, aunque estos \'ultimos estan deprecados en las versiones superiores
a la 2.5.0., y al an\'alisis de especificaciones expresadas en CTL (l\'ogica de \'arbol
computacional) y LTL (l\'ogica temporal lineal).


\item%
\textbf{Arquitectura:}\\


\item%
\textbf{Qualidades de la implementacion:}\\

\emph{NUSMV} esta escrito en ANSI C, y es compatible con POSIX. \emph{NUSMV} usa un avanzado
paquete de BDD desarollado en Colorado University, y provee una interf\'az general para
linking con avanzados resolvedores SAT. Esto produce que NUSMV sea muy robusto, portable,
eficiente, y facil de entender por personas fuera del equipo de desarrollo.\\
\\


\end{itemize}


\subsection{Por qu\'e NuSMV?}

Porque nos da un gran control sobre la eficiencia del checkeo del modelo. Esto se debe
a que ... (Preguntar bien a Pedro o leer mucho para entender :D)


\section{Falluto 2.0}

\subsection{Objetivos}
\begin{itemize}
\item%
Ocultar al usuario las complejidades de tener que definir un sistema usando una herramienta
con la complejidad de NuSMV.
\item
Ocultar las complejidades de definir de manera funcional el comportamiento de fallas,
presentando una sintaxis declarativa para la especificaci\'on de las mismas.
\end{itemize}

\subsection{Implementaci\'on de Falluto 2.0}

Falluto 2.0 esta escrito en Python. Python es un Lenguaje de programación interpretado cuya filosof\'ia hace hincapi\'e en una sintaxis limpia que favorezca un c\'odigo legible.
Se trata de un lenguaje de programaci\'on multiparadigma ya que soporta orientaci\'on a objetos, programaci\'on imperativa y, en menor medida, programaci\'on funcional. Es un lenguaje interpretado, usa tipado din\'amico, es fuertemente tipado y es multiplataforma. Posee una muy completa librer\'ia estadar.
Es administrado por la Python Software Foundation. Posee una licencia de c\'odigo abierto, denominada Python Software Foundation License,1 que es compatible con la Licencia p\'ublica general de GNU a partir de la versión 2.1.1, e incompatible en ciertas versiones anteriores.


\subsubsection{A cerca del valor inicial de actvar\#}

Notar que el valor inicial de esta variable no tiene significado alguno en el
sistema ya que representa la acci\'{o}n que se realiz\'{o} en el estado anterior para 
transitar hasta el actual. Dado que el estado actual es el inicial, osea no hubo
estado anterior, esta variable no representa nada en este momento. Sin embargo
es importante darle un valor correcto para que no afecte la verificaci\'{o}n de
propiedades en el sistema smv resultante.\\

\noindent Las opciones para su valor inicial son:\\
\begin{enumerate}

\item%
    Un valor incial espec\'{i}fico, que solo se use en este caso.
\item%
    Cualquier valor dentro de su dominio.
\end{enumerate}


\newpage % Bibliography goes in new page

\begin{thebibliography}{99}

\bibitem{NuSMVBibl} NuSMV 2.5 User Manual. Roberto Cavada, Alessandro Cimatti, Charles Arthur Jochim, Gavin Keighren,
Emanuele Olivetti, Marco Pistore, Marco Roveri and Andrei Tchaltsev. http://nusmv.fbk.eu/NuSMV/userman/v25/nusmv.pdf

\end{thebibliography}





\end{document}
