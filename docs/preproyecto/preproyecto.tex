\documentclass[12pt]{article}

\usepackage{a4wide}
\usepackage[spanish]{babel}
\usepackage{paralist}

\newcommand{\falluto}{\textsc{Falluto}}

\begin{document}

\section{T\'\i{}tulo}
%

\begin{quote}
  \setlength{\parsep}{1ex}\raggedright\large\sl%
  Revisi\'on y extensi\'on de un model checker para la
  verificaci\'on autom\'atica de sistemas tolerantes a fallas
\end{quote}

\section{Introducci\'on}
 

Un aspecto fundamental para el buen funcionamiento de los distintos
sistemas basados en computadoras (desde peque\~nos sistemas embebidos
a grandes sistemas distribuidos) es el de poder tolerar fallas que
puedan producirse ya sea como consecuencia de la ejecuci\'on misma del
sistema (ej. una divisi\'on por cero) o de agentes externos al sistema
(ej. la p\'erdida de un mensaje de comunicaci\'on).
%
La tolerancia a estas fallas implica que el sistema afectado pueda
recuperarse del efecto que \'esta produce y continuar con su normal
ejecuci\'on.

Un sistema tolerante a fallas es un sistema que incrementa su
confiabilidad respecto a uno que no lo es.
%
O al menos deber\'\i{}a.
%
La incorporaci\'on de la tolerancia a falla en un sistema no es algo
que se logra por mera redundancia de m\'odulos.
%
Un ejemplo paradigm\'atico de un sistema cuyo m\'odulo de tolerancia a
falla no tuvo ning\'un efecto ocurri\'o en el primer lanzamiento del
Ariane~5, el cual no tuvo un desenlace exitoso~\cite{ariane}.

Como cualquier sistema, los sistemas tolerantes a fallas tambi\'en
deben ser verificados.
%
En estos casos, no s\'olo se debe verificar que el sistema preserva
las propiedades de \emph{safety} y \emph{liveness} deseadas, sino
tambi\'en que lo hace a\'un en presencia de fallas.
%
Se han realizado muchos intentos para la verificaci\'on formal de
sistemas tolerantes a fallas, pero la mayor\'\i{}a de los casos se basa
en la incorporaci\'on expl\'\i{}cita del modelado de la falla dentro
del modelo del sistema de una manera ad-hoc.
%
Esta t\'ecnica carece de metodolog\'\i{}a definida y como tal puede
introducir errores en el modelo a verificar.

La incorporaci\'on de las fallas y su tratamiento dentro de un modelo
formal requiere, entonces, un nuevo marco de especificaci\'on, ya sea
para modelar el sistema como para especificar los requerimientos.
%

En este trabajo nos enfocaremos en la verificaci\'on autom\'atica de
sistemas tolerantes a fallas siguiendo el enfoque presentado por el
\emph{model checking}.
%
Model checking~\cite{CGP99,BK2008} es una t\'ecnica de verificaci\'on
que, dado el modelo del sistema bajo estudio y la propiedad requerida,
permite decidir autom\'aticamente si la propiedad es satisfecha o no.

Un primer enfoque a esta soluci\'on fue planteado en~\cite{Hames2009}.
En este trabajo
\begin{inparaenum}[(1)]
\item%
  se introdujo una extensi\'on al lenguaje de NuSMV, denominado
  \falluto, de manera tal de permitir expresar la inyecci\'on de falla
  de manera modular y declarativa, y
\item%
  se implement\'o un \emph{front end} que compilaba esta extensi\'on
  de NuSMV a NuSMV puro y, de producirse un contraejemplo, lo re
  interpretaba en el modelo orginal.
\end{inparaenum}

Este primer trabajo presenta resultados interesantes pero
preliminares.  Hay varios aspectos a revisar:
%
\begin{enumerate}
\item%
  el sublenguaje de NuSMV que forma parte de \falluto\ es
  cr\'\i{}ptico y no permite identificar claramente las transiciones
  de cada m\'odulo;
\item
  m\'as a\'un, este mismo sublenguaje est\'a bajo consideraci\'on para
  entrar en obsolescencia en cuyo caso dejar\'\i{}a de ser mantenido
  en el model checker NuSMV;
\item%
  la extensi\'on de \falluto\ correspondiente a la inyecci\'on de
  fallas contiene instrucciones redundantes y ambiguas;
\item%
  si bien el lenguaje de modelado contempla las caracter\'\i{}sticas
  necesarias para modelar fallas, la l\'ogica de especificaci\'on es
  una l\'ogica temporal tradicional (LTL).
\end{enumerate}


\section{Objetivos}

Este proyecto persigue, entonces, los siguientes objetivos:

\begin{enumerate}
\item%
  La definici\'on de un lenguaje de modelado de sistemas e inyecci\'on
  de fallas que contemple las falencias de \falluto\ y los
  inconvenientes antes mencionado.
\item%
  La definici\'on de una l\'ogica de especificaci\'on que permita
  hablar apropiadamente de tolerancia a fallas.
\item%
  La implementaci\'on de un \emph{front end} para NuSMV que permita la
  verificaci\'on de propiedades dadas en la nueva l\'ogica en modelos
  dados en el nuevo lenguaje.
\end{enumerate}





\section{Metodolog\'\i{}a}

Para la definici\'on del lenguaje de modelado utilizaremos como punto
de partida los conceptos definidos en~\cite{AG1993}
y~\cite{Gartner1998}, que dan un primer paso en la identificaci\'on del
modelado de fallas separ\'andolo del modelo normativo.  El lenguaje
debe describir claramente redes de aut\'omatas con lo cual se deber\'a
parecer m\'as al lenguaje de PRISM~\cite{thesis:parker,prism} que a
NuSMV (aunque sin considerar los saltos probabilistas). Se busca, en
este sentido, una articulaci\'on con el lenguaje definido
en~\cite{Bordenabe2011}.  Tambi\'en ser\'a de gran importancia el
trabajo~\cite{Hames2009} que ya diera un claro primer paso en la
definici\'on de un lenguaje para la especificaci\'on de fallas.

Para la definici\'on de la l\'ogica de especificaci\'on, se
estudiar\'a la l\'ogica dCTL~\cite{CastroEtAl2011}, definida
especialmente para especificar propiedades que expresen tolerancia a
fallas.  Dado su expresividad, se analizar\'a qu\'e subconjunto es
expresable en las l\'ogicas que manipula NuSMV (estas son, CTL, LTL y
PSL).  En base a este estudio se concluir\'a cu\'al es la mejor manera
de proceder con la l\'ogica de especificaci\'on, teniendo en cuenta el
balance entre expresividad y simplicidad del lenguaje respecto de su
complejidad de an\'alisis.

Finalmente, se construir\'a el \emph{front end} para NuSMV que realice
la verificaci\'on de propiedades en la l\'ogica dise\~nada sobre
modelos descriptos en el lenguaje tambi\'en dise\~nado en este
contexto.  Para ello se utilizar\'a el conocimiento adquirido en el
desarrollo del model checker \falluto~\cite{Hames2009}, en particular
lo relacionado al estudio de fairness de fallas.



\section{Plan de trabajo}

Las tareas ha realizar a lo largo de este proyecto incluyen:
%
\begin{enumerate}
\item%
  Revisi\'on bibliogr\'afica, en particular:
  \begin{compactenum}
  \item%
    Estudio del model checker NuSMV.
  \item%
    Estudio del model checker \falluto.
  \item%
    Estudio del lenguaje de modelado de
    \textsc{Offbeat}~\cite{Bordenabe2011}.
  \item%
    Estudio de la l\'ogica dCTL.
  \end{compactenum}
\item%
  Definici\'on del lenguaje de especificaci\'on de comportamiento de
  sistemas tolerantes a fallas
\item%
  Definici\'on de la l\'ogica de especificaci\'on de propiedades de
  tolerancia a fallas.
\item%
  Construcci\'on del \emph{front end} para NuSMV.
\item%
  Documentaci\'on del software realizado.
\item%
  Escritura del manuscrito del trabajo especial.
\end{enumerate}



\bibliographystyle{plain}
\bibliography{biblio}


\end{document}

