% Falluto 2.1 a model checker for fault tolerant systems.
% Raul Monti.
% Monday 7, April 2014.
%==============================================================================

\documentclass[12pt]{article}
\title{Falluto2.1, a model checker for fault tolerant systems.}
\author{ Pedro R. D'Argenio\\ FaMAF - UNC - Argentina \and Ra\'ul E. Monti\\
	 FaMAF - UNC - Argentina
}
\date{\today}


\usepackage{amsmath} % for labelling arrows
\usepackage{verbatimbox} % to be able to center verbatim

\newcommand{\nusmv}{\mbox{\textit{NuSMV}}}
\newcommand{\falluto}{\mbox{\textit{Falluto2.1}}}
\newcommand{\textusc}[1]{$_{\text{#1}}$} % text underscript


%==============================================================================

\begin{document}

\maketitle



\begin{abstract}
Fault tolerant systems are of major importance as part of highly dependable 
systems. Model checking techniques improve our confidence over this kind of
systems throughout exhaustive evaluation of its behaviour. This work presents
a model checker utility named \mbox{\textit{Falluto2.1}} back-ended by 
\mbox{\textit{NuSMV}} symbolic model checker. \mbox{\textit{Falluto2.1}} is
tailored to ease the model checking of fault tolerant systems. Striving for
this is that it offers intuitive and simple mechanisms to work over the 
injection of faults, mechanisms to work over the occurrence of faults under
different fairness arrangements, and mechanisms to guide the checking of 
properties over usual faults behaviour sceneries. \mbox{\textit{Falluto2.1's}}
declarative language makes this model checker attractive to all kind of users,
allowing them to focus on the real behaviour of the system, hiding the implicit
functional aspects.
\end{abstract}


%%%%%%%%%%%%%%%%%%%%%%%%%%%%%%%%%%%%%%%%%%%%%%%%%%%%%%%%%%%%%%%%%%%%%%%%%%%%%%%


\section{Introduction}
Model checkers allow users to model computational systems, usually in an
abstract level, with the aim of exhaustively verifying some property of
interest over them. Thus model checking techniques are considered formal
approaches to software and hardware verification. This approaches are of great
utility in the area of highly dependable systems, as they offer a formal proof
of correctness over the model, or a trace counter-example in case the model
doesn't meet the required properties.\\
A fault in a system can be seen as an unexpected behaviour of its parts, which
can lead it to perform erroneously. Fault tolerant systems are those who
implement mechanisms to overcome the occurrence of faults, avoiding situations
that will take to this undesired performance. As you can imagine, when
implementing highly dependable software or hardware, this kind of systems are
preferred over those who don't implement any kind of fault tolerance.\\
Due to their language generality, actual Model checkers lack of simplicity when
modelling systems under faulty environments, thus making it difficult to test
the designs of fault tolerant systems. Although some other works attempt to
override this issue, they seem to lack of theoretical support over their
mechanisms, and some others don't go very far into the possibilities
of offering syntactic tools for defining a wide range of common scenarios to
work on.\\
This two important matters are issued in our work by means of a simple
declarative language for fault injection, and syntax support for specifying
on one hand the fairness conditions over process and faults in the model, and 
on the other hand the fault behaviour model under which each specific property
will be checked. Furthermore, every result and decision that matter to this
features is supported on well known theoretical background.\\
Finally all of our results have been implemented over a
\mbox{\textit{NuSMV}}\cite{nusmv} model checker front-end, which we have called
\mbox{\textit{Falluto2.1}}.\\
Based on the formal context given by Kripke Structures and Label Transition 
Systems, our model checker \mbox{\textit{Falluto2.1}} allows modelling finite
state systems with deterministic and non-deterministic choices. It is capable of managing
properties in LTL and CTL
logics, as well as it offers meta-properties intended to reduce the work of the
user and avoid the introduction of errors in the modelling of the system.\\
Its main features are its declarative syntax for fault injection and fault
model specification, as well as for process and fault fairness management.\\
Despite the large amount of manipulation that the user defined model receives
in the effort to build the corresponding \mbox{\textit{NuSMV}} model,
\mbox{\textit{Falluto2.1}} features a trace interpreter module, enabling it
to present a user comprehensible counter-example trace when ever needed.

\paragraph{Outline}
The remainder of this article is organized as follows.
Section~\ref{previous work} gives account of previous work.
Our new and exciting results are described in Section~\ref{results}.
Finally, Section~\ref{conclusions} gives the conclusions.


%%%%%%%%%%%%%%%%%%%%%%%%%%%%%%%%%%%%%%%%%%%%%%%%%%%%%%%%%%%%%%%%%%%%%%%%%%%%%%%


\section{Basics}\label{basics}
In this section we introduce some insights in basic concepts that make it
easier to understand our work. We start explaining Kripke Structures and Labelled Transition Systems which will be the formal bases to our analysis of computational systems. We will also introduce \mbox{\textit{NuSMV}} which is our back-end model checker. We will take a look especially at the part of its modelling language which is of our interest. Finally we will introduce the basics of \falluto 's modelling language.\\
But first, some simple conventions we will maintain during this writing:
\begin{itemize}
\item We will be mostly talking about modelling of \textit{finite state systems with deterministic and non-deterministic choices}, unless we explicitly state the opposite.
\item We will call \textit{normal transitions} to those we expect to happen as part of the normative behaviour of the system, while we will call \textit{faulty transitions} to the ones resulting of a modification of the system state by the occurrence of a fault.
\item Systems will be composed from one or more process concurrently running, which will be able to synchronize normal transitions.
\item We consider that we can not predict the instant where faults will occur, and we wont be preventing them from happening, but we can define a fault behaviour model for each property we want to check.
\end{itemize}

\subsection{Kripke structures}\label{kripke} %TODO note that kripke structure modelling is very close to the finall steps of model checking and though we have high control over complexity of the problem.
Kripke Structures are transitions graphs normally used to describe the behaviour of a system. They consist of a set of states, a set of transitions between states, and a function that labels each state with a set of properties that are \textit{true} in this state. \cite{clarke} formally defines a Kripke structure as fallows:\\

Let $AP$ be a set of atomic propositions. A Kripke Structure $M$ over $AP$ is a four tuple $M=(S,S_0,R,L)$ where:
\begin{enumerate}
\item $S$ is a finite set of states
\item $S_0 \subseteq S$ is the set of initial states.
\item $R \subseteq S \times S$ is a transitions relation that must be total, that is, for every state $s \in S$ there is a state $s' \in S$ such that $R(s,s')$.
\item $L : S \rightarrow 2^{AP}$ Is a function that labels each state with the set of atomic propositions \textit{true} in the state.
\end{enumerate}


\subsection{Labeled Transition Systems}\label{lts}
Just as Kripke Structures, Labelled Transitions Systems are transition graphs used for representing computational behaviour. They are formally defined as a four tuple $M = (S,S_{0},L,R)$ where:
\begin{itemize}
\item $S$ is a set of states.
\item $S_0 \subseteq S$ is a set of initial states.
\item $L$ is a set of labels (transitions names).
\item $R \subseteq S \times L \times S$ is a ternary relation (of labelled transitions).
\end{itemize}
In this way, having a three tuple $(s,l,s')$ belonging to $R$, will mean that there is a transition named $l$ coming out from state $s$ and ending in state $s'$.

\subsection{LTS to Kripke}\label{lts2kripke}
Notice that we can translate an LTS system representation to a Kripke equivalent representation in the following way:\\

Let $M_1 = (S_1, S_{1_0} R_1, L_1)$ be the LTS described system, then we can build the Kripke Structure $M_2 = (S_2,S_{2_0},R_2,L_2)$ over the atomic propositions $AP$ in the following way:
\begin{itemize}
\item $ AP = \{action = e ~|~ e \in L_1 \cup \{null\}\} $
\item $ S_2 = S_1 \times (L_1 \cup \{null\}) $
\item $S_{2_0} = S_{1_0} \times \{null\}$
\item $R_2 = \{(s,a) \rightarrow (s',b) ~|~ s\overset{b}{\rightarrow}s' \in R_1, a \in L \cup \{null\}\}$
\item $L_2(s,a) = (action = a),$ for all $(s,a) \in S_2$
\end{itemize}
What we have just done is to build for each state $s_i$ and label $e$ in the LTS a state in the Kripke structure that represents the fact of arriving to the state $s_i$ using the label $e$. Given that at the begging of the executions we don't do any action to get to the initial state, is that we have also defined for each state $s \in S_{1_0}$ a state $(s,null)$ that represents it in $S_2$. It is pretty intuitive the way that we build the relation function. The labelling indicates which action was taken to get to each state, and combined with the name of the state, uniquely identifies which transition is being taken in the original LTS description of the system.


\subsection{NuSMV symbolic model checker}\label{nusmv}
\nusmv is a symbolic model checker originated from the reengineering,
reimplementation and extension of SMV, the original BDD-based model checker
developed at CMU~\cite{cimatti}. We will be using its second version which
integrates model checking techniques based on propositional satisfiability
(SAT) to the BDD-based symbolic model checking of its original version.\\
\nusmv ~language allows us to describe process as to be modules.\footnote{Note
that we will only be describing those aspects of \nusmv ~language that are
important to our work.} Each module
can be composed of three sections, one describing state variables, an other
restricting the initial states set, and a last one defining the relation or
transition function from each state to the following. Modules can then be run
concurrently from a main module who would be the head of execution of the whole
described system.\\
The initial states of a module are described by a boolean formula over the
state variables of the module. In a similar manner, but considering not
only the actual valuation of the variables but also the valuations in
the next state, we can define the transition relation by a boolean formula.
In this sense a the transition relation will define a correlation between
valuations in the origin states and the reaching states.\\
We will pay attention to four types of variables in \nusmv, namely \textit{int}, \textit{boolean}, \textit{user enumerated}, and \textit{arrays}, all of whose domains will be of course
finite, because of the finite state restriction of our models.\\
%TODO we could replace this for a bigger example and place it in the appendix. Or have both maybe.
The following is an example of a \nusmv ~module representing a car:

\begin{verbatim}
MODULE car
    VAR
        state: {engineON, engineOF}
        fuel: 0..3
    INIT
        state = engineOF & fuel = 3
    TRANS
        ( estate = engineON & fuel <= 0 ) 
        -> 
        ( next(estate) = engineOF )
\end{verbatim}

Note that any pair of states valuations, one taken for the actual state and the
other for the next state, whom evaluate the transition formula to \emph{true},
will be in the transition relation. That is for example
$$(\{state=engineOF, fuel=3\},\{state=engineON,fuel=3\}) \in R$$ where $R$ is our
transition relation of the car model.\\
It is important to notice that the close relation between \nusmv module declarations and the corresponding formal description of a process in terms of a Kripke Structure. First we notice that the set of states in the Kripke Structure will be determined by the \texttt{VAR} section of the module, where each combination of states variables valuation will represent a different state. The initial set of states will be given by the \texttt{INIT} formula, who will restrict it to only those valuations that evaluate in \textit{true} this formula. Finally the transition relation will be given by the combination of \textit{actual} and \textit{next} states whose valuations satisfy the formula in the \texttt{TRANS} section.\\
\nusmv language also supports unconditional and strong conditional fairness constraints over the paths to be checked. This are identified by the keywords \texttt{FAIRNESS} for unconditional fairness and \texttt{COMPASSION} for strong conditional fairness.\\
Finally LTL and CTL properties can be introduced using keywords \texttt{LTLSPEC} and \texttt{CTLSPEC} followed by the corresponding formula to be checked over the model.


%%%%%%%%%%%%%%%%%%%%%%%%%%%%%%%%%%%%%%%%%%%%%%%%%%%%%%%%%%%%%%%%%%%%%%%%%%%%%%%


\section{Falluto2.1's syntax}

\begin{verbbox}
PROCTYPE counter ( ; count )
    VAR
        clock:0..5

    INIT
        clock = 0

    TRANS
        [count]: => count' = (count+1)%6

ENDPROCTYPE

INSTANCE counter1 = counter(count1)
INSTANCE counter2 = counter(count2)
INSTANCE counter3 = counter(count2)
INSTANCE counter4 = counter(count1)

\end{verbbox}

\begin{figure}[h]
  \centering
  \theverbbox
  \caption{\falluto normative behaviour modelling example.}
  \label{example}
\end{figure}

\subsection{System normative behaviour modelling}
Fallowing the same direction showed for \nusmv \texttt{MODULES} syntax, our process descriptions in \falluto find a clear representation of the Kripke Structures that define the system.\\
While depicting the normative behaviour, each process description is again divided in three sections: \texttt{VAR}, \texttt{INIT} and \texttt{TRANS}. This are enclosed by the keywords \texttt{PROCTYPE} and \texttt{ENDPROCTYPE}.\\
While sections \texttt{VAR} and \texttt{INIT} mantain the same structure and meaning as in \nusmv syntax, it is in \texttt{TRANS} where we find a real change in the syntax. As we can see in Figure \ref{example}, we won't be describing the transitions relation as a boolean formula, but we will describe each transitions separately, with the possibility of giving each one a name. In fact each transition will have the structure
\begin{center}
\texttt{[name]: enabling\_condition => effects}
\end{center}
meaning that from each state where \texttt{enabling\_condition} evaluates to \textit{true} there exists a transition named \texttt{name} to a new state with equal valuation except for the changes made explicit in \texttt{effects}.\\
Several \texttt{PROCTYPES} will describe the behaviour of the different components of our system. This \texttt{PROCTYPES} should be instantiated as many times as we like, resulting in the interliving components of our modelled system. We do this using \falluto's keyword \texttt{INSTANCE} as shown in Figure \ref{example}.\\
Synchronization is worked out by shearing the same synchronization names while making the instantiation. In Figure \ref{example} we can see that \texttt{conter1} and \texttt{counter4} synchronize their transitions \texttt{count}, as they share the same synchronization name \texttt{count1}. So do \texttt{conter2} and \texttt{counter3} using \texttt{count2} name.\\
Although we won't get to much into it, \falluto offers mechanisms for being able to read other processes state variables, as well as for making global definitions over state variables and constants.

\subsection{Fault injection syntax}
As one of the main advantages of \falluto, we find its declarative language for fault injection. It offers us the possibility of injecting in a simple line of code practically all the information that describes the behaviour of a fault over our processes. The three types of faults actually supported are \textit{stop} faults, \textit{transient} faults and \textit{bizantine} faults, each of which we will define next.
\begin{itemize}

\item \textbf{Stop faults:} are those which ones occurred permanently disable one or more transitions in a process. They usually correspond to the breakage of a component of the modelled process. We use the following one line syntax
\begin{center}
\texttt{name: enable\_cond => effects is STOP(act1,\ldots,actN)}
\end{center}
to indicate that a fault named \texttt{name} could occur if \texttt{enable\_cond} evaluates to \textit{true} in the actual state, producing over the next state the changes described at \texttt{effects}, and from now on permanently disabling in  this process transitions with names in \{\texttt{act1}, \ldots, \texttt{actN}\}.

\item \textbf{Transient faults:} transient faults are those which occur infinitely often. They affect the system in the instant they occur, but their effects are not permanent. They usually correspond to transient faults produced by an unfortunate environment, as could be a bit flip in information passing through a large network. We use the following one line syntax
\begin{center}
\texttt{name: enable\_cond => effects is TRANSIENT}
\end{center}
to indicate that the transient fault named \texttt{name} could occur in this process each time \texttt{enable\_cond} evaluates to \textit{true}, having the effects described at \texttt{effects}.

\item \textbf{Byzantine faults:} Byzantine faults are those that usually represent a malfunction over some system component. That could be a bad memory cell that looses its charge from time to time. They are then linked to specific state variables. We use the following one line syntax
\begin{center}
\texttt{name: enable\_cond => effects is BYZ(var1,\ldots,varN)}
\end{center}
to indicate that the fault named \texttt{name} could occur when ever \texttt{enable\_cond} evaluates to \textit{true}, producing the instant effects in \texttt{effects}, and permanently affecting the variables in (\texttt{var1},\ldots,\texttt{varN}). Thus from the occurrence of this fault and on, variables in (\texttt{var1},\ldots,\texttt{varN}) could at any time change their values to another value inside their range, as a permanent effect of this fault.
\end{itemize}
Faults are defined inside \falluto's proctypes, in a new section named \texttt{FAULT}. They can be referenced in the rest of the model just as we reference any normal transition by using the keyword \texttt{just()}. An example can be found at Figure \ref{fullexample}.

\subsection{Properties specification}
\falluto uses \nusmv capabilities for model checking properties in \textit{LTL} and \textit{CTL} logics. That is, we can use any of this logics to check our modelled system. Notice that the atomic propositions to be checked when using this formulas are not only conformed by valuations of state variables, but we can also talk about normal actions and faults inside our properties. This is, using the keyword \texttt{just}, we can talk about an action as to be a property of any state, in the sense of having reached that state via this action. For example
\begin{center}
\texttt{LTLSPEC G F just(some\_action)}
\end{center}
would be an LTL property used to check that transition \texttt{some\_action} is often taken in our modelled system, given that \texttt{just(some\_action)} would evaluate to \textit{true} in each state that we reach by taking the transition named \texttt{some\_action}, and \textit{false} in every other state.\\
A battery of meta-properties offered by \falluto allows the user to easily define some typical fault occurrence scenarios where to check its properties. This mechanisms are thought to avoid the introduction of large amount of modifications over the model to be able to make such checking. Thus, it avoids the introduction of errors, makes the modelling simpler, and specifies a fixed methodology for model checking fault tolerant systems.\\
The present meta-property battery is compose by the following:
\begin{itemize}
\item \textbf{Normal behaviour meta-property:} We use this meta-property to check some LTL or CTL property over executions lacking of fault occurrences. Thus it would be like model checking an equivalent system where no faults have been defined.
\begin{itemize}
\item \textit{syntax: } \texttt{NORMAL\_BEHAVIOUR -> q}
\item \textit{semantic: } if no faults occur then property \texttt{q} is \textit{true}.
\end{itemize}
\item \textbf{Finitely many faults meta-property:} We use this meta-property to check some LTL or CTL property over executions where faults finally stop occurring.
\begin{itemize}
\item \textit{syntax: } \texttt{FIN\_MANY\_FAULTS(f1,...,fm) -> q}
\item \textit{semantic: } if faults in \texttt{\{f1,...,fm\}} finally stop happening, then property \texttt{q} is \textit{true}.
\end{itemize}
\item \textbf{Ensure meta-property:} we use this meta-property to check some LTL or CTL property over executions where we ensure lapses of stability in the system.
\begin{itemize}
\item \textit{syntax: } \texttt{ENSURE (N,a1,a2,\ldots,am) WITHOUT (f1,f2,\ldots,fk) -> q}
\item \textit{semantic: } if often actions in \texttt{\{a1,a2,\ldots,am\}} are ensure to happen for a lapse of \texttt{N} transitions without faults from \texttt{\{f1,f2,\ldots,fk\}} happening in between, then property \texttt{q} is \textit{true}.
\end{itemize}
\item \textbf{Atmost meta-property:} Thought to be used to check some limited tolerance to fault occurrence.
\begin{itemize}
\item \textit{syntax: } \texttt{ATMOST (N,f1,f2,\ldots,fm) -> q}
\item \textit{semantic: } if faults in \texttt{\{f1,f2,\ldots,fm\}} occur at most \texttt{N} times, then property \texttt{q} is \textit{true}.
\end{itemize}
\end{itemize}

%%%%%%%%%%%%%%%%%%%%%%%%%%%%%%%%%%%%%%%%%%%%%%%%%%%%%%%%%%%%%%%%%%%%%%%%%%%%%%%


\section{Fairness}\label{fairness}

Usually when checking properties over a model we are interested in analysing 
executions that don't differ with what would be a realistic behaviour of the
real system that it represents. That is the case for example of a reliable
communication channel in the sense that we don't expect it to continuously
loose a message, or when modelling a fair scheduler we don't expect it to
infinitely ignore a process that is trying to be attended.\\
Fairness constraints can be seen as conditions over the traces to be checked
during the model checking process. This conditions ensure that the traces %TODO is it true???
to be considered go pass states with a certain property infinitely often. In
this sense we can look at the model as a generalized Buchi automaton where %TODO is it important to look at it as a Buchi Automaton???
the fairness conditions define an acceptance set of the automaton.\\
It is of our interest to distinguish between three kinds of fairness
constraints, to be:
\begin{itemize}

\item \textbf{Unconditional fairness:} Given a property \texttt{p} over the
set of states of our system, an unconditional fairness in relation to
\texttt{p} will restrict our analysis to those executions where we often pass
through any state in the set of states where \texttt{p} evaluates to
\textit{true}. If we look at our system as a generalized Buchi automaton
$\beta=(Q,\Sigma,\delta,q_{0},F)$, then adding this constraint will correspond
to a new automata $\beta'=(Q,\Sigma,\delta,q_{0},F\cup F')$ where $F'$ is the
set of all states where \texttt{p} evaluates to \textit{true}. 

\item \textbf{Strong fairness:} Given properties \texttt{p} and \texttt{q}
over the set of states of our system, a strong fairness constraint with
condition \texttt{p} in relation to \texttt{q} will restrict our analysis
to those executions where if eventually property \texttt{p} is continuously
evaluated to \textit{true} then property \texttt{q} should often evaluate
to \textit{true}. %TODO how can I look it as a Buchi automaton (if possible)

\item \textbf{Weak fairness:} Given properties \texttt{p} and \texttt{q}
over the set of states of our system, a weak fairness constraint with
condition \texttt{p} in relation to \texttt{q} will restrict our analysis
to those executions where if property \texttt{p} is often evaluated to 
\textit{true} then property \texttt{q} should also often evaluate to
\textit{true}. %TODO how can I look it as a Buchi automaton (if possible)

\end{itemize}


\subsection{Process fairness}\label{process fairness}
In \falluto we would want to be able to talk about the transitions we take. This will enable us to name them inside properties we want to model check or even use them to specify fairness between process.\\
We have seen in section \ref{lts2kripke} how we can take the names of transitions in a LTS representation, and look at them as properties of the reached states in a Kripke structure equivalent representation of the system. Based on this fact, we have developed a simple mechanism to represent just taken transitions inside our \nusmv modules. This allows us to step over a state and talk about the transition just taken to get there.\\
Being able to talk about actions will enable us to easily define fairness
conditions over the execution of the process that live inside our system.\\
Using this kind of fairness conditions we can restrict our model
checking analysis to those execution traces of interest, where each process
can execute some action fairly often, avoiding situations where for example a
selective group of process are the only cause of the system progress.\\        %TODO fixme->selective
%As we can see, fairness constraints are a kind of conditions to the execution
%paths we are going to check. In this sense they are conditioning the system
%scheduler that organizes the parallel composition of the process involved in
%the model.\\
\falluto implements by defect, although it can be disabled, the following weak fairness
condition for each process \texttt{P} in the model. Making use of the
underlying \nusmv capabilities for specifying unconditional
fairness constraints, we define for each process \texttt{P} in the system the
weak fairness property:
$$FAIRNESS \bigvee_{t \in T_{N_{P}}} ex_t \vee deadlock_P$$
Here deadlock indicates that no normal transition of the process is enabled. On the other hand, $ex_t$ will be a formula that evaluates to \textit{true} in states where we have just reached by normal transition $t$ of process $P$, understanding as normal transition those that are not the result of a faulty behaviour.\\
Notice that a process can go out from a deadlock state through the
changes made by the rest of the composed system. Thus this is
indeed a weak fairness constraint, as to be enabled often is not enough for a
process to ensure it will be attended. On the other hand if the process never
falls in deadlock, this constraint ensures that the process will eventually
make a normal transition.


\subsection{Fault fairness}\label{fault fairness}
In \mbox{\textit{Falluto2.1}} we go further on in this concept and also 
stablish fairness conditions between faults and process. This is, in what
concerns to fairness executions, we look at faults as individual process
that also struggle to be attended by the scheduler.\\
\mbox{\textit{Falluto2.1}} implements by defect, although it can be disabled, an unconditional fairness
constraint over the set of \textit{normal transitions} of the system. That
is, again making use of the underlying \mbox{\textit{NuSMV}} capabilities for
specifying unconditional fairness constraints, we define the following
constraint to work with: $$FAIRNESS \bigvee_{t \in T_N} ex_t \vee deadlock$$
Notice that this constraint implies that in the case where finally there is always some
normal transition enabled to be taken, i.e. when finally the system stops falling in deadlock, then the system should often take one.
Thus using this we are avoiding the analysis of traces where the occurrence of faults take
control over the progress of the system. 


\section{Semantics and compilation in \falluto}\label{compilation}

In this section we will go into the core of \falluto. We will attempt to give an overview of the compilation from a \falluto syntax defined model into a \nusmv model, backing decisions and results on Kripke Structures theory.\\
When compiling into the \nusmv language, we will only build one \textit{main} \nusmv module. Remember that this \textit{main} module is head of execution when model checking with \nusmv. The system state will be represented by the joining of all state variables of each \textit{instanced proctype} defined in the original model. Though the \textit{VAR} section of our compiled module will be composed from this states variables, as well as from an extra variable which we will call the \textit{action\_variable}. This last variable will indicate for each state, the name of the transition taken to get there. Thus it is simply the result of the translation of the LTL description given by names in \falluto's transitions to the Kripke structure description forced by our selected \nusmv syntax.\footnote{ See section \ref{lts2kripke} } This allows us to easily compile \falluto's \texttt{just()} actions to propositions over the value of \textit{action\_varible}. This is we compile \texttt{just(action)} to \texttt{action\_variable = action}.\\
The \textit{initialization} section of the \textit{main} module will be built using the \textit{and} composition of the initialization of each of the \textit{instanced proctypes}. This will evidently accomplish the expected initial state of the composed system.\\
Finally the \textit{transition} section will be built by the \textit{or} composition of each compiled normal and faulty transition, indicating that in a first approach the composed system can select any of them to make its transitions.\\
In what rests of this section we will explain first the compilation of normal transitions, next that from the different kind of faults, and finally we will show how we compile our meta-properties and how they modify the behaviour of the compiled model.\\
If we look at how we will compile our transitions, we will note that the value of the \textit{action\_variable} and the values of the program counters will make them exclusively satisfiable. This is will not be able to combine two of them and still get a satisfiable formula. Thus when we consider that the transition formula will be an \textit{or} composition of each of them, it becomes clear that only one member of the composition will become \textit{true} in each transition and thus only the modifications stated at the original transition definition will be made to the next state.

\subsection{Normal transitions compilation}
A normal transition
\begin{center}
\texttt{[name]: enable => effects}
\end{center}
will have the following compilation scheme:
\begin{center}
\texttt{!stop\_faults \& next(action\_variable) = name \& enable\textusc{compiled} \& effects\textusc{compiled} \& pc = i \& dont\_change}
\end{center}
meaning that all \textit{stop faults} which disable this transition should not be active, the last action in the next state should correspond to the name of this transition, the enabling conditions of the transitions should be met, and so the resulting effects over the arriving state. \texttt{pc=i} ,where \texttt{pc} is system variable for the specific instance who is making the transition, allows us to distinguish between different transitions with same name inside the same instance. Thus \texttt{i} is ,inside this instance, simply an unique id associated to this transition. Note that this program counter variables (\texttt{pc}) will be part of the variables section of the \nusmv \textit{main} module. Finally \texttt{dont\_change} states that every variable except from \texttt{action\_variable} and those modified at \texttt{effects\textusc{compiled}}, should remain with its present value. 
\subsection{Faults compilation}
Faults are also compiled as possible transitions to make.
\subsubsection*{STOP faults}
\textit{STOP} faults like
\begin{center}
\texttt{name: enable => effects is STOP(a1,...,aN)}
\end{center}
will be compiled with the following scheme:
\begin{center}
\texttt{!this\_fault\_active \& next(this\_fault\_active) \& next(action\_variable) = name \& enable\textusc{compiled} \& effects\textusc{compiled} \& dont\_change }
\end{center}
meaning that, as this faults are permanent, they should not already be active and they should be active from next state and on, the action in next state should be set to this faulty transition, and both \texttt{enable} conditions and \texttt{effects} changes should be met. Note that \texttt{this\_fault\_active} is a boolean variable that indicates if the fault is active, and will be also present in the variables section of the main module and set to \textit{true} at the initialization section formula.

\subsubsection*{BYZANTINE faults}
Each time we find a Byzantine fault we will compile it into several transitions. That is, one for the occurrence of the fault, and one for each Byzantine effect it will keep on having over the affected variables.\\
A Byzantine fault of the form
\begin{center}
\texttt{name: enable => effects is BYZANTINE(var1,...,varN)}
\end{center}
Will be compiled in one transition with the scheme
\begin{center}
\texttt{!this\_fault\_active \& next(this\_fault\_active) \& next(action\_variable) = name \& enable\textusc{compiled} \& effects\textusc{compiled} \& dont\_change}
\end{center}
and \textit{N} transitions with the scheme
\begin{center}
\texttt{this\_fault\_active \& next(action\_variable) = biz\_effect \& next(varX) in range \& dont\_change}
\end{center}
representing the effects over the Byzantine broken variables. That is, if the Byzantine fault that produces faulty effects over variable \texttt{varX} for some \texttt{X} between 1 and \textit{N} is active, then, taking this transition, \texttt{varX} will change its value to some random value in its range.

\subsubsection*{TRANSIENT faults}
Transient faults are not permanent, thus they lack of activation variable. While compiling a transient fault of the form
\begin{center}
\texttt{name: enable => effects is TRANSIENT}
\end{center}
we will get an only transition
\begin{center}
\texttt{next(action\_variable) = name \& enable\textusc{compiled} \& effects\textusc{compiled} \& dont\_change}
\end{center}


\subsection{Properties compilation}
We will focus on the compilation of each of our meta-properties, as common user defined properties don't imply a change on the rest of the compiled module.

\subsubsection*{Normal Behaviour meta-property compilation}
Compiling this kind of properties suggests modifying the behaviour of the model, avoiding faults from occurring. Thus given the \falluto original property
\begin{center}
\texttt{NORMAL\_BEHAVIOUR -> q}
\end{center}
We modify our transition formula \textit{T} to get a new formula 
\begin{center}
\textit{T' = T \& !(next(action\_variable) in \{f\textusc{1},f\textusc{2},...,f\textusc{N}\})}
\end{center}
where \textit{f\textusc{1}} to \textit{f\textusc{N}} name all the faults defined in our model, thus avoiding ever stepping over a faulty transition. Note that we reducing the reachable states and cutting of faulty executions, making the problem smaller.
Finally we just compile \texttt{q} as a common LTL or CTL property to check over this modified model.

\subsubsection*{Finitely many faults meta-property compilation}
In this case we are looking to stop the occurrence of faults from some future state and on. Looking for this, we define a new boolean variable called \textit{stop\_faults} at our state variables section, initializing it to \textit{false} at the initialization section, and we proceed like fallows. Given the finitely many fault formula
\begin{center}
\texttt{FINITELY\_MANY\_FAULTS(f\textusc{1},\ldots,f\textusc{N}) -> q}
\end{center}
we replace our transition formula $T$ for a new one $T' = T ~\&~ Q$ where
\begin{verbatim}
Q = (!stop_faults) | ( stop_faults
                     & next(stop_faults)
                     & !(next(action_variable) in {f1,...,fN})
                     )
\end{verbatim}
%Lets make the following definitions\\
%\begin{itemize}
%\item $T = T_1|T_2$
%\item $T_1 = t_1 | ... | t_k$ with the $t_i$ all representing normal transition, i.e. they are of the form $(next(action\_var) = nt_i~\&~t'_i)$ with $nt_i$ the name of a normal transition 
%\item $T_2 = t_{k+1} | ... | n_n$ with the $t_i$ all representing faulty transitions, i.e. they are of the form $(next(action\_var) = f_i~\&~t'_i)$ with $f_i$ the name of a fault
%\item $Q = Q_1 | Q_2$
%\item $Q_1 = !stop\_faults$
%\item $Q_2$ = the right side of $Q$ formula.
%\item And then we get\\
%$T' = T~\&~Q = (T_1|T_2)~\&~(Q_1|Q_2) = (Q_1~\&~(T_1|T_2)) ~|~ ((Q_2~\&~T_1) | (Q_2~\&~T_2))$
%\item and with\\
%$T'_1 = (Q_1~\&~(T_1|T_2))$\\
%$T'_2 = ((Q_2~\&~T_1) | (Q_2~\&~T_2))$\\
%we get $T' = T'_1 | T'_2$
%\end{itemize}
Notice that $stop\_faults$ value is only being modified at $Q$, and that as we already said it is initialized to \textit{false}. Now lets look at our original transition formula as 
$T = T_1 | T_2$ with $T_1 = t_1 | ... | t_k$ all representing normal transitions of our system, i.e. $t_i$ with $i=1..k$ are of the form $(next(action\_var) = nt_i~\&~t'_i)$ with $nt_i$ the name of a normal transition, and $T_2 = t_{k+1} | ... | t_N$ all representing faulty transitions, i.e. $t_i$ with $i=k+1..N$ are of the form $(next(action\_var) = f_i~\&~t'_i)$ with $f_i$ the name of a fault. Lets name $Q_1 =~!stop\_faults$ and lets name $Q_2$ the other side of $Q$ formula, so that $Q = Q_1 | Q_2$.
Then we get that $T' = T ~\&~ Q = (T_1|T_2) \& (Q_1|Q_2) = (Q_1~\&~(T_1|T_2)) ~|~ ((Q_2~\&~T_1) | (Q_2~\&~T_2))$. Lets define $T'_1 = (Q_1~\&~(T_1|T_2))$ and $T'_2 = ((Q_2~\&~T_1) | (Q_2~\&~T_2))$ so that $T' = T'_1 | T'_2$. Now lets take a look at our new transition formula. When we start our executions, \textit{stop\_faults} is set to \textit{false} thus clearly $T' = T'_1$ defines our transitions. As there is no rule in $T'_1$ over the next value of \textit{stop\_faults}, its next value will be non-deterministically assigned. Also note that $Q_1$ = \textit{true} at this point and thus $T' = T'_1 = T_1 | T_2 = T$ and the model behaves just as the original one, with the exception that now it has the new variable $stop\_faults$ which value is modified arbitrarily at each transition made.\\
If $stop\_faults$ suddenly gets the value \textit{true} then $T' = T'_2$ and given that $Q_2$ specifies that $next(action\_variable)$ shouldn't take a fault name as value, we get that the formula $(Q_2~\&~ T_2)$ becomes not satisfiable and thus $T'_2 = (Q_2 ~\&~ T_1)$, and from now on the system will make only normal transitions.\\
As we want to test $q$ over execution where $stop\_faults$ finally occurs, we for last compile a \nusmv property with the formula
\begin{center}
\texttt{LTLSPEC~ (F ~stop\_faults) -> q}\\
or\\
\texttt{CTLSPEC ~(AF~ stop\_faults) -> q}
\end{center} 
accordingly to which formal logic $q$ has been specified.

\subsubsection*{Ensure meta-property compilation}
Given the ensure property $$ENSURE(N,a_1,...,a_n)~WITHOUT(f_1,...,f_m)~\rightarrow~q$$
we introduce the integer variable $count\#$ into our model, which will allow to count the length of the ``safe'' lapses. This variable is initialized to $N$, which is the lapses lengths we want to ensure.\\
Given our original transition formula $T$ we build a new one $T' = T \& Q$, where $Q$ is as fallows:\\\\
$ (~(count\# = N~\&~(next(count\#)=N~|~next(count\#)=0))$\\
$ |~(count\# < N~\&~next(action\_variable)~in~\{a_1,...,a_n\}~\&$\\
$~~~~next(count\#)=count\#+1)$\\
$ |~(count\# < N~\&~!(next(action\_variable)~in~\{a_1,...,a_n,f_1,...,f_m\}~\&$\\
$~~~~next(count\#)=count\#)$\\
$ )$\\
Suppose that $count\#$'s value is $N$, just as when the execution starts. Then the model execution has to take the first option at $Q$, and this can either execute as usually kipping the value of $count\#$ as $N$, or change it to $0$. Once it changes it to $0$ it enters the ``safe'' zone, and can only take second and third options in $Q$, meaning that it will be unable to make a faulty action in $\{f_1,...,f_m\}$ until $count\#$ becomes $N$ again, which will happen after taking $N$ times the second option in $Q$.\\
As we want to check $q$ property under the assumption that we often ensure a ``safe'' lapse, we need to condition our check property to often entering the ``safe'' zone, i.e. when $count\#$ becomes $0$.\\
Thus our property to check over this modified model will be
\begin{center}
\texttt{LTLSPEC (G~F~count\#=0) -> q}\\
or\\
\texttt{CTLSPEC (AF~count\#=0) -> q}
\end{center}
depending on the formal logic of $q$

\subsubsection*{Atmost meta-property compilation}
Given the \textit{ATMOST} meta-property $$ATMOST(N,f_1,...,f_m) \rightarrow q$$ we introduce a new integer variable $count\#$ ranging between $0$ and $N$, which will serve as a counter for the occurrence of faults in $\{f_1,...,f_m\}$. We initialize $count\#$ in $0$ and change or original transition formula $T$ for a new formula $T' = T~\&~Q$ with\\
$Q = $\\ 
$~~~~~~(~(~count\# < N~\&~next(action\_variable)~in~\{f_1,...,f_m\}$\\
$~~~~~~~~~\&~next(count\#)=count\#+1)$\\
$~~~~~~|~(~count\# < N~\&~!(next(action\_variable)~in~\{f_1,...,f_m\})$\\
$~~~~~~~~~\&~next(count\#)=count\#)$\\
$~~~~~~|~(~count\# = N~\&~!(next(action\_variable)~in~\{f_1,...,f_m\})$\\
$~~~~~~~~~\&~next(count\#)=count\#)$\\
$~~~~~~)$\\\\
Notice that once we have made $N$ faulty transitions from the set $\{f_1,...,f_m\}$, $count\#$'s value is $N$ and we are forced to take third option in $Q$, which ensures that we won't take any of those faulty transitions again.\\
Then it only rests to check property $q$ over this modified model.


\section{Conclusion}

%%%%%%%%%%%%%%%%%%%%%%%%%%%%%%%%%%%%%%%%%%%%%%%%%%%%%%%%%%%%%%%%%%%%%%%%%%%%%%%


%\bibliographystyle{abbrv}
%\bibliography{main}
\begin{thebibliography}{9}

\bibitem{clarke}
    Edmund M. Clarke, Jr., Orna Grumberg, and Doron A. Peled. 2000.
    \emph{Model Checking}. MIT Press, Cambridge, MA, USA. 

\bibitem{cimatti}   
    Alessandro Cimatti, Edmund M. Clarke, Enrico Giunchiglia, Fausto
    Giunchiglia, Marco Pistore, Marco Roveri, Roberto Sebastiani, and Armando
    Tacchella. 2002. \emph{NuSMV 2: An OpenSource Tool for Symbolic Model
    Checking.} In \emph{Proceedings of the 14th International Conference on
    Computer Aided Verification} (CAV '02), Ed Brinksma and Kim Guldstrand
    Larsen (Eds.). Springer-Verlag, London, UK, UK, 359-364. 

\end{thebibliography}

\end{document}

%==============================================================================